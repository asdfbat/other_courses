\documentclass[12p,a4paper]{article}
\usepackage[utf8]{inputenc}
\usepackage[T1]{fontenc,url}
\usepackage{multicol}
\usepackage{multirow}
\usepackage{parskip}
\usepackage{lmodern}
\usepackage{microtype}
\usepackage{verbatim}
\usepackage{amsmath, amssymb}
\usepackage{tikz}
\usepackage{physics}
\usepackage{mathtools}
\usepackage{algorithm}
\usepackage{algpseudocode}
\usepackage{listings}
\usepackage{enumerate}
\usepackage{graphicx}
\usepackage{float}
\usepackage{hyperref}
\usepackage{tabularx}
\usepackage{siunitx}
\usepackage{fancyvrb}
\usepackage[makeroom]{cancel}
\usepackage[margin=2.4cm]{geometry}
\renewcommand{\baselinestretch}{1}
\renewcommand{\exp}{e^}
\renewcommand{\b}{\boldsymbol}
\newcommand{\h}{\hat}
\newcommand{\m}{\mathbb}
\newcommand{\half}{\frac{1}{2}}
\renewcommand{\exp}{e^}
\setlength\parindent{0pt}


\begin{document}
\title{STK1110 -- Oblig 1}
\author{
    \begin{tabular}{r l}
        Jonas Gahr Sturtzel Lunde & (\texttt{jonassl})
    \end{tabular}}
% \date{}    % if commented out, the date is set to the current date

\maketitle

\hspace{10cm}

\section*{Oppgave 2}
\subsection*{a)}
Ettersom ligning (1) fra oppgaven er en t-fordeling med $n-1$ frihetsgrader, vet vi at den følger
\begin{equation}
    P\qty(t_{\alpha/2,\, n-1} < \frac{\bar{X} - \mu}{S/\sqrt{n}} < t_{1-\alpha/2,\, n-1}) = 1-\alpha
\end{equation}
der $t_{\alpha/2,\, n-1}$ og $t_{1-\alpha/2,\, n-1}$ er $\alpha/2$ og $1-\alpha/2$ persentilene til en t-fordeling med $n-1$ frihetsgrader. 

Løser vi ulikheten inni parantesen for $\mu$ får vi at
\begin{equation*}
    \bar{X} - t_{\alpha/2,\, n-1}\cdot\frac{S}{\sqrt{n}} < \mu < \bar{X} + t_{1 - \alpha/2, n-1}\cdot\frac{S}{\sqrt{n}}
\end{equation*}
som er $100(1-\alpha)\%$ konfidensintervallet til $\mu$.


\subsection*{b}
Ettersom ligning (1) fra oppgaven er kjikvadrat-fordelt med $n-1$ frihetsgrader, vet vi at den tilfredsstiller
\begin{equation}
    P\qty(\chi_{\alpha/2,\, n-1} < \frac{(n-1)}{\sigma^2}S^2 < \chi_{\alpha/2,\, n-1}) = 1 - \alpha
\end{equation}
der $\chi_{\alpha/2,\, n-1}$ og $\chi_{1-\alpha/2,\, n-1}$ er $\alpha/2$ og $1-\alpha/2$ persentilene til en kjikvadrat-fordeling med $n-1$ frihetsgrader. 

Løser vi ulikheten inni parantesen for $\sigma$ får vi at
\begin{equation}
    \sqrt{\frac{(n-1)}{\chi_{\alpha/2,\, n-1}}}S < \sigma < \sqrt{\frac{(n-1)}{\chi_{1-\alpha/2,\, n-1}}}S
\end{equation}




\section*{Oppgave 3}
\subsection*{a)}
\begin{equation*}
    F(x) = \int\limits_{-\infty}^x f(x) \dd{x} = \int\limits_{\kappa}^x \theta\kappa^\theta x^{-\theta-1} \dd{x} = \qty[\theta\kappa^\theta\frac{x^{-\theta}}{-\theta}]_\kappa^\theta = 1 - \qty(\frac{\kappa}{x})^\theta
\end{equation*}


\subsection*{b)}
\begin{equation}
    F(x) = 1 - \qty(\frac{\kappa}{x})^\theta = \half \ \Rightarrow \ \frac{\kappa}{x} = \frac{1}{2}
\end{equation}







\end{document}
