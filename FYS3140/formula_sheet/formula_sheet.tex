\documentclass[10pt,a4paper]{article}
\usepackage[utf8]{inputenc}
\usepackage[T1]{fontenc,url}
\usepackage{parskip}
\usepackage{lmodern}
\usepackage{microtype}
\usepackage{verbatim}
\usepackage{amsmath, amssymb}
\usepackage{tikz}
\usepackage{physics}
\usepackage{mathtools}
\usepackage{algorithm}
\usepackage{algpseudocode}
\usepackage{listings}
\usepackage{enumerate}
\usepackage{enumitem}
\usepackage{graphicx}
\usepackage{float}
\usepackage{hyperref}
\usepackage{tabularx}
\usepackage{siunitx}
\usepackage{multicol}
\usepackage{multirow}
\usepackage{framed}
\usepackage{fancyvrb}
\usepackage{xcolor}
\usepackage[a4paper, margin=0.2cm]{geometry}

\renewcommand{\baselinestretch}{1}
\renewcommand{\b}{\textbf}
\renewcommand{\exp}{e^}

\newcommand{\infint}{\int_{-\infty}^{\infty}}
\newcommand{\zeroinfint}{\int_{-\infty}^{\infty}}

\newcommand{\infsum}{\sum_{n=-\infty}^{\infty}}
\newcommand{\zeroinfsum}{\sum_{n=0}^{\infty}}
\newcommand{\oneinfsum}{\sum_{n=1}^{\infty}}

\newcommand{\holine}{\rule{286pt}{1pt}}
\newcommand{\Holine}{\rule{286pt}{3pt}}

\newcommand{\half}{\frac{1}{2}}

\newcommand{\gr}{\colorbox{green}}
\newcommand{\yl}{\colorbox{yellow}}

\setlength{\columnsep}{0.4cm}
\setlength{\columnseprule}{0.06cm}
\setlength{\FrameSep}{0.2cm}





\begin{document}
\begin{multicols}{2}


\section*{Usefull Shit}
\subsection*{Random shit I always forget}
\[
    c \ln{x} = \ln{x^c} \quad\quad\quad \dv{x}\ln{x} = \frac{1}{x}
\]
\\
\[
    x = \frac{-b\pm \sqrt{b^2 - 4ac}}{2a}
\]
\subsection*{Integration}
\[
    \int (uv') = uv - \int(u'v)
\]


\holine
\subsection*{Trigenometric Identities}
\[
    \exp{\pm i z} = \cos{z} \pm i\sin{z} \\
\]
\[
    \cos{z} = \half \qty(\exp{iz} + \exp{-iz}) \quad\quad
    \sin{z} = \frac{1}{2i}\qty(\exp{iz} - \exp{-iz})
\]


\holine
\subsection*{Common ODE solutions}
\textbf{Harmonic oscilator}
\[
    u''(z) = -\omega^2 u(z)
\]
\[
    u(z) = k_1 \cos(\omega z) + k_2\sin(\omega z) = c_1 \exp{i\omega x} + c_2 \exp{-i\omega x}
\]


\section*{Complex analysis}
Ask Julie about principal value thingy.





% ██████╗ ██████╗ ███████╗
%██╔═══██╗██╔══██╗██╔════╝
%██║   ██║██║  ██║█████╗  
%██║   ██║██║  ██║██╔══╝  
%╚██████╔╝██████╔╝███████╗
% ╚═════╝ ╚═════╝ ╚══════╝
\newpage
\section*{Ordinary Differential Equations}
% \begin{tikzpicture}[level distance=2.5cm,
%   level 1/.style={sibling distance=8cm},
%   level 2/.style={sibling distance=6cm},
%   level 3/.style={sibling distance=3cm}]
%     \node[circle, draw](z){$ODEs$}
%         child{node[circle,draw]{1st Order}
%             child{node[circle,draw]{Separable}}
%             child{node[circle,draw]{Linear}}
%             }
%         child{node[circle,draw]{2nd Order}
%             child{node[circle,draw]{Homogeneous}
%                 child{node[circle,draw]{Constant coeff.}}
%                 child{node[circle,draw]{Euler-Cauchy}}
%                 }
%             child{node[circle,draw]{Inhomogeneous}
%                 child{node[circle,draw]{Constant coeff.}}
%                 child{node[circle,draw]{Factorization}}
%                 child{node[circle,draw]{Variation of parameters}}
%                 }
%             };
% \end{tikzpicture}




\subsection*{First Order, Linear, ODEs - Integrating Factor}
\[
    y'(x) + P(x) y(x) = Q(x)
\]

\[
    y(x)\mu(x) = \int Q(x)\mu(x) \dd{x} + C \quad\quad\text{with}\quad\quad \mu(x) = \exp{\int P(x)\dd{x}}
\]




\Holine
\section*{Homogenous ODEs}
\[
    y'' + P(x)y' + Q(x)y = 0
\]
\textbf{Properties}
\begin{itemize}
    \item Linear combination of solutions is also a solution
    \item General solution on form $y(x) = c_1 y_1(x) + c_2 y_2(x)$
    \item Linearly independent solutions have a Wronskian of 0.
\end{itemize}
\[
    W(x) = 
    \begin{vmatrix} y_1(x) & y_2(x) \\ y_1'(x) & y_2'(x) \end{vmatrix}
\]


\holine
\subsection*{Variation of the constant}
If you have one of the two linearly indepedent solutions, you can find the other as $y_2(x) = C(x)\cdot y_1(x)$, where $C(x)$ is a functions determined by inserting $y_2(x)$ into the ODE.

When you arrive at a solution for $C(x)$, you may discard any constants or coefficients, i.e. $C(x) = \alpha x^3 + \beta$.


\holine
\subsection*{Constant coefficients - Particular Equation}
\[
    y''(x) + ay'(x) + by(x) = 0
\]

Solve the particular equation
\[
    \lambda^2 + a\lambda + b = 0
\]
for $\lambda_1$ and $\lambda_2$.


\subsubsection*{Two, real roots}
\[
    y(x) = C_1\exp{\lambda_1 x} + C_2\exp{\lambda_2 x}
\]


\subsubsection*{One, real root}
\[
    y(x) = (C_1 + xC_2)\exp{\lambda x}
\]


\subsubsection*{Two, complex roots}
\begin{equation*}
\begin{split}
    y(x) = A\exp{\lambda_1 x} + B\exp{\lambda_2 x} = \exp{-a/2 x}\qty[A\exp{i\omega x} + B\exp{-i\omega x}] \\
    =\exp{-a/2 x}\qty[\hat{A}\cos{\omega x} + \hat{B}\sin{\omega x}]
\end{split}
\end{equation*}


\holine
\subsection*{Euler-Cauchy Equations}
\[
    x^2y'' + a_1x y' + a_0 y = 0
\]

Introducing
\[
    x = \exp{z} \quad\quad\Rightarrow\quad\quad z = \ln|x|
\]
The equation can be rewritten to
\[
    \pdv[2]{y}{z} + (a_1-1)\pdv{y}{z} + a_0 y = 0
\]
Solve and insert for $z$.



\holine
\subsection*{Power methods}
\begin{itemize}
    \item Represent $P(x)$ and $Q(x)$ as power series.
    \item Assume solution on the form
    \begin{itemize}
        \item $y(x) = \sum_{n=0}^\infty a_n x^n$
        \item $y'(x) = \sum_{n=1}^\infty n a_n x^{n-1}$
        \item $y''(x) = \sum_{n=2}^\infty n(n-1) a_n x^{n-2}$ 
    \end{itemize}
    \item Insert back into ODE.
    \item Split into equations of matching powers of $x$.
\end{itemize}
This will give you one or two undetermined coefficients. The equations maybe give the coefficients as a series depending on each other, like $a_{s+1} = a_s^2$. If the series are odd/even, two undetermined coefficients are required to describe them, so the solution is complete.



\holine
\subsection*{Fröbenius method}
\[
    x^2 y'' + xb(x)y' + c(x) y = 0
\]
Assuming solution on form
\[
    y_p(x) = \sum_{m=0}^\infty a_m x^{m+s}
\]
where $s$ is some real number determined  the \textit{indicial equation}
\[
    s(s-1) + b_0s + c_0 = 0
\]
where $b_0=b(0)$, and $c_0=c(0)$

\textbf{Three possible scenarios:}
\begin{itemize}
    \item Different roots, $s_1\neq s_2$, and $s_1-s_2\neq \text{integer}$.
    \begin{itemize}
        \item Two indepedens solutions $y_i(x) = x^{si}\sum_{m=0}^\infty a_0 x^m$
    \end{itemize}
    \item Different roots, $s_1\neq s_2$, but $s_1-s_2=\text{integer}$. ($s_1>s_2$).
    \begin{itemize}
        \item Solve for both by Power Series.
        \item Often, $s_2$ gives the complete solution (two undetermined coefficients), so try this first.
        \item Sometimes, only $s_1$ gives a solution. Find the other by variation of the constant.
    \end{itemize}
    \item Double root, $s_1 = s_2$.
    \begin{itemize}
        \item Find the solution by Power Series.
        \item Find the second solution by variation of the constant.
    \end{itemize}
\end{itemize}



\Holine
\section*{Inhomogenous ODEs}
\[
    y'' + P(x)y' + Q(x)y = R(x)
\]
\textbf{Remember} to always rewrite to this form.

\textbf{Properties}
\begin{itemize}
    \item Solutions on form\\
    $y(x) = y_h(x) + y_p(x) = c_1y_1(x) + c_2y_2(x) + y_p(x)$.
    \item $y_h(x)$ is the solution to the homogenous equation.
    \item $y_p(x)$ is \textit{any} solution to the whole ODE.
    \item Since $y_h$ contains two arbritrary constants, $y_p$ should contain none.
\end{itemize}



\holine
\subsection*{Inhomo ODEs with constant coefficients}
\[
    y'' + ay' + by = R(x)
\]
\begin{itemize}
    \item Make a guess at $y_p$ with the same form as $R(x)$, with unknown coefficients.
    \item Insert back into ODE to solve for coefficients.
\end{itemize}
\textbf{Special case:} $R(x) = A\exp{kx}$.\\
Let $\alpha$ and $\beta $ be the roots of $\lambda^2 + a\lambda + b = 0$.
\begin{enumerate}
    \item If $k\neq \alpha,\beta$: Try $y_p = C\exp{\lambda x}$.
    \item If $k = \alpha \text{ or } \beta$: Try $y_p = Cx\exp{\lambda x}$.
    \item If $k = \alpha = \beta$: Try $y_p = C\exp{\lambda x}$.
\end{enumerate}


\holine
\subsection*{Inhomo ODEs with varying coefficients}
\[
    y'' + P(x)y' + Q(x)y = R(x)
\]

\subsubsection*{Factorization}
If $u(x)$ is a known solution to the homo-ODE, a particular solution is $y_p = u(x)\cdot v(x)$ where
\[
    w' = v \quad\quad\quad w' + \qty[\frac{2u'}{u} + P]w = \frac{R}{u}
\]
Solve the ODE for $w$ with integrating factor.


\subsubsection*{Variation of parameters}
\[
    y_p = -y_1\int\frac{y_2R}{W}\dd{x} + y_2 \int\frac{y_1R}{W}\dd{x}
\]
where $y_1$ and $y_2$ are known linearly indepedendent solutions to the homo-ODE.
\textbf{NOTE:} Remember that $R(x)$ is the RHS after the ODE is rewritten on the standard form.








%███████╗ ██████╗ ██╗   ██╗██████╗ ██╗███████╗██████╗ 
%██╔════╝██╔═══██╗██║   ██║██╔══██╗██║██╔════╝██╔══██╗
%█████╗  ██║   ██║██║   ██║██████╔╝██║█████╗  ██████╔╝
%██╔══╝  ██║   ██║██║   ██║██╔══██╗██║██╔══╝  ██╔══██╗
%██║     ╚██████╔╝╚██████╔╝██║  ██║██║███████╗██║  ██║
%╚═╝      ╚═════╝  ╚═════╝ ╚═╝  ╚═╝╚═╝╚══════╝╚═╝  ╚═╝
\newpage

\section*{Trigonometric Functions}
\subsection*{Orthogonality}
\[
    \int_{-\pi}^\pi \sin(mx)\sin(nx) \dd{x} = \int_{-\pi}^\pi \cos(mx)\cos(nx) \dd{x} = \pi \delta_{mn}
\]


\holine
\section*{Fourier Series}
\[
    f(x) = \half a_0 + \oneinfsum a_n\cos(\frac{n\pi x}{L}) + \oneinfsum a_n\sin(\frac{n\pi x}{L})
\]
\[
    a_n = \frac{1}{L}\int\limits_{-L}^{L}f(x) \cos(\frac{n\pi x}{L}) \dd{x} \quad\quad
    b_n = \frac{1}{L}\int\limits_{-L}^{L}f(x) \sin(\frac{n\pi x}{L}) \dd{x}
\]
\[
    f(x) = \sum_{n=-\infty}^{\infty} c_n \exp{in\pi x/L}  \quad\quad
    c_n = \frac{1}{2L}\int_{-L}^L f(x) \exp{-in\pi x/L}
\]

\subsection*{Even and Odd functions}
If $f(x)$ is \b{even} $\qty[f(x) = f(-x)]$:
\[
    a_n = \frac{2}{L}\int_0^L f(x)\cos(\frac{n\pi x}{L}) \dd{x} \quad\quad b_n = 0
\]
If $f(x)$ is \b{odd} $\qty[f(x) = -f(-x)]$:
\[
    a_n = 0 \quad\quad b_n = \frac{2}{L}\int_0^L f(x)\sin(\frac{n\pi x}{L}) \dd{x}
\]


\holine 
\subsection*{Dirichlet Conditions for Fourier Series}
\begin{enumerate}
    \item Finite number of min/max in interval.
    \item Finitite number of (only) finite discontinuities.
\end{enumerate}
If this holds, then the series will converge to $f(x)$ at all points. At discontinuities, the series will converge to the mid-point.
\holine
\subsection*{Parseval's Theorem}
\[
    \int_{-L}^L |f(x)|^2 \dd{x} = 2L\sum_{-\infty}^\infty |c_n|^2
\]



\holine
\section*{Fourier Transforms}
\[
    f(x) = \frac{1}{\sqrt{2\pi}} \infint F(k) \exp{ikx} \dd{k} \quad\quad
    F(k) = \frac{1}{\sqrt{2\pi}} \infint f(x) \exp{-ikx} \dd{x}
\]

\subsection*{Odd and even functions}
If $f(x)$ is an odd function, $f(x) = -f(-x)$, the Fourier transform can be done using only sine (as cosine is symmetric around 0):
\[
    f(x) = \sqrt{\frac{2}{\pi}} i\int\limits_0^{\infty} F(k) \sin(k x) \dd{k}   \quad
    F(k) = \sqrt{\frac{2}{\pi}} i\int\limits_0^{\infty} f(x) \sin(k x) \dd{x}
\]
If $f(x)$ is even, $f(x) = f(-x)$, we need only cosine (as sine is anti-symmentric):
\[
    f(x) = \sqrt{\frac{2}{\pi}} \int\limits_0^{\infty} F(k) \cos(k x) \dd{k}   \quad
    F(k) = \sqrt{\frac{2}{\pi}} \int\limits_0^{\infty} f(x) \cos(k x) \dd{x}
\]


\subsection*{FT of a derivative}
\[
    \mathcal{F}\qty[f^{(n)}(x)] = (ik)^n \mathcal{F}\qty[f(x)]
\]


    




%██████╗ ██████╗ ███████╗
%██╔══██╗██╔══██╗██╔════╝
%██████╔╝██║  ██║█████╗  
%██╔═══╝ ██║  ██║██╔══╝  
%██║     ██████╔╝███████╗
%╚═╝     ╚═════╝ ╚══════╝
\newpage
\section*{Partial Differential Equations}
\subsection*{Notes}
\begin{itemize}
    \item In symentrical systems, it seems your can switch x<->y if it is required to suit boundary conditions (example: Diritchlet conditions are at x=a instad of at y=b).
    \item When resulting in cos/sin solutions of frequencies, include n=0 for cos, as it gives a non-zero solution, but not for sin.
\end{itemize}

\holine
\subsection*{Separation of variables}
1) The the solution function as a product of functions of each variable, i.e:
\[
    u(x,y) = X(x)Y(y) \quad\quad\quad u(r, \theta) = R(r)T(\theta)
\]
2) Insert this into the equation, and separate the equation into parts of only each variable. Each side must then be constant, and equals some \textit{separation constant}.

3) Solve each side of the equation (equaling the seperation constant), giving an infinite set of \textit{eigenfunctions}, $u_n(x,y)$ for the equation.

4) The final solution is a linear combination of the eigenfunctions.
\[
u(x,y) = \sum_{n=-\infty}^\infty c_n u_n(x,y)
\]



\holine
\subsection*{Laplace Equation - 2D Cartesian}
\[
    \nabla u(x,y) = 0
\]

Separation of variables, $u(x,y) = X(x)Y(y)$ gives solutions
\[
    u(x,y) = X(x)Y(y) = \qty{\begin{matrix} \exp{ky} \\ \exp{-ky} \end{matrix}} \times \qty{\begin{matrix} \sin(kx) \\ \cos(kx) \end{matrix}}
\]

\textbf{Diritchlet BC:} $u(x,0) = u(0,y) = u(a,y) = 0$, $u(x,b) = f(x)$
Eigenfunctions on form
\[
    u_n(x,y) = A_nF_n(x)G_n(x) = A_n\sin(n\pi x/a)\sinh(n\pi y/a)
\]


\holine
\subsection*{Wave Equation}
\[
    \pdv[2]{y}{x} = \frac{1}{c^2}\pdv[2]{y}{t}
\]
Seperation of variables, $u(x,t) = F(x)G(t)$ gives equations
\[
    F''(x) = -k^2F(x) \quad\quad\quad \ddot{G}(t) = -k^2c^2G(t)
\]
where the seperation constant, $-k^2$ must be negative, else the solutions would blow up.

The equations has solutions on the form
\[
    y(x,t) = \qty{\begin{matrix} \sin(kx) \\ \cos(kx) \end{matrix}} \times \qty{\begin{matrix} \sin(kct) \\ \cos(kct) \end{matrix}}
\]
The end-points are usually fixed at 0, leaving only the $\sin(kt)$ term, and forcing $k=n\pi/L$.

If the velocity is 0 at $t=0$, we discard the sin-velocity term and get

\[
    y(x,t) = \oneinfsum b_n\sin(\frac{n\pi x}{L})\cos(\frac{n\pi c t}{L})
\]

If position is 0 at $t=0$, we discard the cos-velocity term instead.

Initial position will be on the form
\[
    y(x,0) = \oneinfsum b_n \sin(\frac{n\pi x}{L}) = f(x)
\]
where $f(x)$ is the initial position or velocity. The coefficients $b_n$ are now Fourier coefficients, given as:
\[
    b_n = \frac{1}{2L}\int_{0}^{L} f(x)\sin(\frac{n\pi x}{L}) \dd{x}
\]



\holine
\subsection*{Orthogonal Functions}
Functions on the form
\[
    p(x)y'' + p'(x)y' + \qty[q(x) + \lambda r(x)]y = 0
\]
on some interval $[a,b]$ has solutions as linear combinations of eigenfunctions $y_n(x)$ which are orthogonal with respect to $r(x)$ such that
\[
    \int_a^b r(x)y_n(x)y_m(x)^* \dd{x} = 0 \quad\quad \text{for} \lambda_n \neq \lambda{m}
\]
Any function can be written as a linear combination of these eigenfunctions
\[
    f(x) = \sum_{n=1}^\infty a_n y_n(x)
\]
then the set $\{y_n(x)\}$ is complete. The coefficients $a_n$ are determined by the orthogonality:
\[
    a_n = \int_a^b f(x)r(x)y_n(x)^* \dd{x}
\]


\newpage





\end{multicols}
\end{document}
