\documentclass[10pt,a4paper]{article}
\usepackage[utf8]{inputenc}
\usepackage[T1]{fontenc,url}
\usepackage{parskip}
\usepackage{lmodern}
\usepackage{microtype}
\usepackage{verbatim}
\usepackage{amsmath, amssymb}
\usepackage{tikz}
\usepackage{physics}
\usepackage{mathtools}
\usepackage{algorithm}
\usepackage{algpseudocode}
\usepackage{listings}
\usepackage{enumerate}
\usepackage{enumitem}
\usepackage{graphicx}
\usepackage{float}
\usepackage{hyperref}
\usepackage{tabularx}
\usepackage{siunitx}
\usepackage{multicol}
\usepackage{multirow}
\usepackage{framed}
\usepackage{fancyvrb}
\usepackage{xcolor}
\usepackage{titlesec} % For title spacing.
\usepackage[a4paper, margin=0.2cm]{geometry}

\renewcommand{\baselinestretch}{1}
\renewcommand{\b}{\textbf}
\renewcommand{\exp}{e^}

\newcommand{\infint}{\int_{-\infty}^{\infty}}
\newcommand{\zeroinfint}{\int_{-\infty}^{\infty}}

\newcommand{\infsum}{\sum_{n=-\infty}^{\infty}}
\newcommand{\zeroinfsum}{\sum_{n=0}^{\infty}}
\newcommand{\oneinfsum}{\sum_{n=1}^{\infty}}

\newcommand{\holine}{\rule{286pt}{1pt}}
\newcommand{\Holine}{\rule{286pt}{3pt}}

\newcommand{\half}{\frac{1}{2}}


% Colorcoding. First for inside equations, second for text.
% Yellow for "problem".
\newcommand{\yl}[1]{\colorbox{yellow}{$\displaystyle #1$}}
\newcommand{\yll}{\colorbox{yellow}}
% Green for "solution".
\newcommand{\gr}[1]{\colorbox{green}{$\displaystyle #1$}}
\newcommand{\grr}{\colorbox{green}}
% Blue for "additional info".
\newcommand{\bl}[1]{\colorbox{cyan}{$\displaystyle #1$}}
\newcommand{\bll}{\colorbox{cyan}}


\setlength{\columnsep}{0.4cm}
\setlength{\columnseprule}{0.06cm}
\setlength{\FrameSep}{0.2cm}

% Setting section spacing
\titlespacing\section{0pt}{0pt}{0pt}
\titlespacing\subsection{0pt}{0pt}{0pt}
\titlespacing\subsubsection{0pt}{0pt}{0pt}




\begin{document}
\begin{multicols}{2}


\section*{Usefull Shit}

\begin{equation*}
    \yl{
        c \ln{x} = \ln{x^c} \quad\quad\quad \dv{x}\ln{x} = \frac{1}{x}
    }
\end{equation*}


\subsection*{Random shit I always forget}
\[
    c \ln{x} = \ln{x^c} \quad\quad\quad \dv{x}\ln{x} = \frac{1}{x}
\]
\\
\[
    x = \frac{-b\pm \sqrt{b^2 - 4ac}}{2a}
\]
\subsection*{Integration}
\[
    \int (uv') = uv - \int(u'v)
\]


\holine
\subsection*{Trigenometric Identities}
\[
    \exp{\pm i x} = \cos{x} \pm i\sin{x} \\
\]
\[
    \cos{x} = \half \qty(\exp{ix} + \exp{-ix}) \quad\quad
    \sin{x} = \frac{1}{2i}\qty(\exp{ix} - \exp{-ix})
\]
\[
    \sinh{z} = \half\qty(\exp{z}-\exp{-z}) \quad\quad \cosh{z} = \half\qty(\exp{z}+\exp{-z})
\]

\holine
\subsection*{Common ODE solutions}
\textbf{Harmonic oscilator}
\[
    u''(z) = -\omega^2 u(z)
\]
\[
    u(z) = k_1 \cos(\omega z) + k_2\sin(\omega z) = c_1 \exp{i\omega x} + c_2 \exp{-i\omega x}
\]




% ██████╗ ██████╗ ███╗   ███╗██████╗ ██╗     ███████╗██╗  ██╗
%██╔════╝██╔═══██╗████╗ ████║██╔══██╗██║     ██╔════╝╚██╗██╔╝
%██║     ██║   ██║██╔████╔██║██████╔╝██║     █████╗   ╚███╔╝ 
%██║     ██║   ██║██║╚██╔╝██║██╔═══╝ ██║     ██╔══╝   ██╔██╗ 
%╚██████╗╚██████╔╝██║ ╚═╝ ██║██║     ███████╗███████╗██╔╝ ██╗
% ╚═════╝ ╚═════╝ ╚═╝     ╚═╝╚═╝     ╚══════╝╚══════╝╚═╝  ╚═╝
\newpage
\section*{Complex analysis}
\subsection*{Usefull shit}

\begin{itemize}
    \item $(z-z_0)<R$ means all complex numbers within radius $R$ of $z_0$ in the complex field.
    \item In many functions, the order of it's pole is very obvious. i.e $1/(z-3)$ is a first order pole at $z=3$, and $1/(z+2i)^3$ is a third order pole at $z=-2i$.
    \item When encountered by a fraction with $i$ in the denominator, multiply by the complex conjugate to move the $i$ upstairs. (i.e. $1/(3+2i)$, multiply by $(3-2i)$). In general:
    \[
        (x+iy)(x-iy) = (x^2 + y^2)
    \]
\end{itemize}

\[
    \ln{z} = \ln|z| + i\theta, \quad\quad \theta\in[-\pi,\, \pi]
\]



\holine
\subsection*{Polar representation and roots}
\[
    z = x + iy = r(\cos(\theta) + i\sin(\theta)) = r\exp{i\theta}
\]

\textbf{Powers of z:}
\[
    z^n = (r\exp{i\theta})^n = r^n\exp{in\theta} = \cos(n\theta) + i \sin(n\theta)
\]

\textbf{Roots of z:} 
\[
    z^{1/n} = r^{1/n}\exp{i(\theta + 2\pi k)/n}, \quad\quad\quad k\in 0,1,2,...,n-1
\]
$z^{1/n}$ has $n$ roots, spread evenly in a circle in the complex plane.



\holine
\subsection*{Complex Series}
The complex sequence
\[
    \{z_n\} = \{z_1, z_2, z_3, ...\}
\]
converges if both the real and imaginary parts of $z_n$ approaches zero for large $n$.

The complex series
\[
    s_n = \sum_{k=1}^n z_k
\]
converges if $z_k$ converges.

\textbf{Ratio test:}
if $\frac{z_{n+1}}{z_n} \leq 1$ for large $n$, then $z_k$ converges.


\holine
\subsection*{Complex Power Series}
\[
    \sum_{n=0}^\infty a_n(z-z_0)^n
\]
Around a point $z_0$, series converges for the area of $z$ where
\[
    |z-z_0| < \lim_{n\rightarrow\infty} \qty|\frac{a_n}{a_{n+1}}| = R
\]
where $R$ is called the \textit{radius of convergence}.


\holine
\subsection*{Analytic Functions}
Analytic functions are special in that they treat $z=x+iy$ as a single unit, i.e. respect the complex structure.

If the output can be expressed solely in $z$ (without $x$ or $y$), the function is analytic. Remember that $x = \half (z+z^*)$ and $=\frac{1}{2i}(z-z^*)$.

An function analytic in a region always has an unique derivative in the region.

\textit{Regular point:} Point where $f$ is analytic.\\
\textit{Singular point:} Point where $f$ is not analytic.

If $f$ is analytic in some region (has a first derivative), it has derivatives of all orders in that region.



\subsubsection*{Cauchy-Riemann Equations}
\[
    \pdv{u}{x} = \pdv{u}{y} \quad\quad \pdv{u}{y} = - \pdv{v}{x}
\]
Criteria for a function to be analytic in a region, derived from demanding existence of the derivative.



\holine
\subsection*{Harmonic Functions}
Harmonic functions are solutions to the \textbf{2D Laplace equation}:
\[
    \nabla^2 \phi = \pdv[2]{\phi}{x} + \pdv[2]{\phi}{y} = 0
\]

If $f(z) = u(x,y) + iv(x,y)$ is analytic in some region, then $u(x,y)$ and $v(x,y)$ are harmonic functions.

\textbf{Theorem:} Given a harmonic function $u(x,y)$, we can always find it's \textit{harmonic conjugate} $v(x,y)$ such that $f = u + iv$ is an analytic function.



\holine
\subsection*{Contour Integrals of Complex Functions}
\[
    \int_\Gamma f(z) \dd{z}
\]


\subsubsection*{Upper Bound Estimate of Contour Integral}
\[
    \qty|\int_\Gamma f(z) \dd{z}| \leq M\cdot L
\]
where $M$ is the maximum value of $f(z)$ on $\Gamma$, and $L$ is the length of $\Gamma$.


\subsubsection*{Independence of Path}
If $\Gamma_1$ and $\Gamma_2$ are two contours that can be continously deformed into one another (without crossing singularities), then
\[
    \int_{\Gamma_1} f(z) \dd{z} = \int_{\Gamma_2} f(z) \dd{z}
\]

\textbf{Cauchy's Theorem:} As a result, any contour integral that doesn't enclose a singularity, is 0.


\holine
\subsection*{Cauchy's Integral Formula}
Formula for evaluating the contour integral around a $n+1$'th order pole at $z_0$.
\[
    \int_\Gamma \frac{f(z)}{(z-z_0)^{n+1}}\dd{z} = \frac{2\pi i}{n!}f^{(n)}(z_0)
\]
\textbf{Note:} Rewrite the expression until it is on the form above. If the contour contains several singularities, rewrite the above expression to handle each of the poles seperately.



\holine
\subsection*{Taylor Series}
\[
    f(z_0) = \sum_{n=0}^\infty a_n(z-z_0)^n, \quad\quad a_n = \frac{f^{(n)}(z_0)}{n!}
\]
\textbf{Theorem:} If $f(z)$ is analytic in the disk $|z-z_0| \leq R$, then the Taylor series converges for all $z$ \textit{inside} the disk.


\holine
\subsection*{Laurent Series}
We combine the \textit{Taylor} series with a \textit{Principal} series of negative powers.
\[
\yl{
    f(z_0) = \sum_{n=0}^\infty a_k(z-z_0)^n - \sum_{n=1}^\infty b_k\frac{1}{(z-z_0)^n}
}
\]
\begin{itemize}
    \item The Taylor series of positive powers converge \textit{inside} some circle $|z-z_0| < R_2$.
    \item The Principal series of negative powers converge \textit{outside} some circle $R_1 < |z-z_0|$.
    \item The Laurent series converges in the donut between the two circles, $R_1 < |z-z_0| < R_2$.
\end{itemize}
\textbf{Tip:} If you only need the series to converge outside/inside some circle, you only need one of the series.

The factor $b_0$ is called the \textbf{residue} of $f$ at $z_0$.

\subsection*{Finding Laurent Series}
\begin{itemize}
    \item If the Laurent Series should expand from a point $z_0 \neq 0$, you must make a substitution $w = z - z_0$.
    \item Manipulate the function to the form
    \[
        f(w) = C(w)\cdot \frac{1}{1-g}
    \]
    where $g$ is any factor/power of $w$, and $C(w)$ is any function of $w$.
    \item The Laurent Series is then given as
\[
\gr{
    f(w) = C(w)\cdot \frac{1}{1-g} = \begin{cases} \ \ C(w)\sum\limits_{n=0}^\infty w^n \quad\quad \text{(Taylor)} \\ -C(w)\sum\limits_{n=1}^\infty\dfrac{1}{w^n} \quad\quad \text{(Principal)} \end{cases}
}
\]
\end{itemize}



\holine
\subsection*{Singularities}
Assume $f(z)$ has an isolated singularity at $z_0$, and it's Larent series is as given above.
\begin{itemize}
    \item If all $b_n = 0$, $z_0$ is a \textit{removable} singularity (not actually a singularity).
    \item If $b_n \neq 0$ for some $n$, but zero for all factors above $n$ (such that $(z-z_0)^{-n}$ is the biggest negative power), we say that $z_0$ is a \textit{pole} of order $n$.
    \item If there are infinite negative terms, we say that $z_0$ is an \textit{essential} singularity.
\end{itemize}


\holine
\subsection*{Residue Theory}
Any integral over a contour $\Gamma$ can be split up into integrals over only infinitesimally small contours around all singularities in $\Gamma$.

An contour integral containing $N$ singularities $z_k$ is given as the sum of the residues at all the singularities.

\[
    \int_\Gamma f(z) \dd{z} = 2\pi i \sum_{k=1}^N Res(f,z_k)
\]

\subsection*{Ways of finding residues}
\begin{itemize}
    \item \textbf{\underline{Use Laurent Series} (always works):} Write out the Laurent Series of the expression around the singularities, and find the $b_1$ term (the $1/z$ coefficient).
    \item \textbf{\underline{For Simple Poles} (alt 1):} \\
    $\gr{Res(f, z_0) = \lim\limits_{z\rightarrow z_0}(z-z_0)f(z)}$\\
    \item \textbf{\underline{For Simple Poles} (alt 2):}\\
    If $f$ is a rational function $f(z) = \frac{P(z_0)}{Q(z_0)}$:\\
    $\gr{Res(f,z_0) = \frac{P(z_0)}{Q'(z_0)}}$
    \item \textbf{\underline{For Multiple Poles}:} If $f$ has a pole of order $m$ at $z_0$, and $M\geq m$, then\\
    $\gr{Res(f,z_0) = \lim\limits_{z\rightarrow z_0} \frac{1}{(M-1)!} \dv[M-1]{z}\qty[(z-z_0)^M f(z)]}$
\end{itemize}




% ██████╗ ███████╗ █████╗ ██╗         ██╗███╗   ██╗████████╗
% ██╔══██╗██╔════╝██╔══██╗██║         ██║████╗  ██║╚══██╔══╝
% ██████╔╝█████╗  ███████║██║         ██║██╔██╗ ██║   ██║   
% ██╔══██╗██╔══╝  ██╔══██║██║         ██║██║╚██╗██║   ██║   
% ██║  ██║███████╗██║  ██║███████╗    ██║██║ ╚████║   ██║██╗
% ╚═╝  ╚═╝╚══════╝╚═╝  ╚═╝╚══════╝    ╚═╝╚═╝  ╚═══╝   ╚═╝╚═╝
\newpage
\section*{Applications to Real Integrals}
\subsection*{Type I: Trigonometric integrals over $[0, 2\pi]$}
\[
    \int_0^{2\pi} u(\cos{\theta}, \sin{\theta}) \dd{\theta}
\]

Substitute for

\[
\yl{
    \cos{\theta} = \dfrac{1}{2}\qty(z+\dfrac{1}{z}) \quad\quad \sin{\theta} = \dfrac{1}{2i}\qty(z-\dfrac{1}{z}) \quad\quad \dd{\theta} = \dfrac{\dd{z}}{iz}
}
\]

The integral is now around a circular contour in the complex plane, centered around $(0,0)$ with radius $1$. Evaluate the integral by finding singularities inside the circle and solving for residues.




\holine
\subsection*{Type IIa: Rational Functions Over $[-\infty, \infty]$}
\[
    I = \int_{-\infty}^\infty f(x) \dd{x} = \int_{-\infty}^\infty \dfrac{P(x)}{Q(x)} \dd{x}
\]

\[
    I = 2\pi i \sum Res(f, z_n)
\]
where $z_n$ are the singularities in the \textit{upper} plane.


Works if
\begin{itemize}
    \item $degree(Q) \geq degree(P) + 2$
    \item $f$ is analytic on and above the complex plane.
\end{itemize}



\holine
\subsection*{Jordan's Lemma}
If $m>0$ is real, and $P$ and $Q$ are polynomials such that $P/Q$ is rational, then:
\[
    \lim\limits_{\rho\rightarrow\infty}\int\limits_{C_\rho} \frac{P(z)}{Q(z)}\exp{imz} \dd{z} = 0
\]
where $C_\rho$ is a half-circle contour with radius $\rho$



\holine
\subsection*{Type IIb: ... with Trigonometric Functions}
\[
    I = \int_{-\infty}^\infty \dfrac{P(x)}{Q(x)}\cos(m x) \dd{x} \quad\quad \text{(or sin)}
\]



\holine
\subsection*{Type III: Singularities on the real axis (Principal Value)}
When a real integral passes singularities, we say that the integral is not defined, but it's \textbf{principal value} is. It behaves just as an ordinary integral:
\[
\yl{
    PV\int_a^b f(x) \dd{x} = \lim\limits_{r\rightarrow 0^+} \qty[\int\limits_a^{C-r} f(x) \dd{x} + \int\limits_{C+r}^b f(x) \dd{x} ]
}
\]

where $C$ is a singularity. Using the same logic as in Type II, with an added infinite half-circle on the upper plane, this evaluates to
\[
\gr{
    PV\infint f(x) \dd{x} = 2\pi i \sum_k Res(f; z_k) + \pi i \sum_j Res(f;z_j)
}
\]
where $\bl{z_k}$ are singularities in the \bll{upper half plane}, and $\bl{z_j}$ are singularities \bll{\textit{on} the real axis.}

We see that singularities on the exit contribute \textit{half} of those above.




% ██████╗ ██████╗ ███████╗
%██╔═══██╗██╔══██╗██╔════╝
%██║   ██║██║  ██║█████╗  
%██║   ██║██║  ██║██╔══╝  
%╚██████╔╝██████╔╝███████╗
% ╚═════╝ ╚═════╝ ╚══════╝
\newpage
\section*{Ordinary Differential Equations}

\subsection*{First Order, Linear, ODEs - Integrating Factor}
\[
    y'(x) + P(x) y(x) = Q(x)
\]

\[
    y(x)\mu(x) = \int Q(x)\mu(x) \dd{x} + C \quad\quad\text{with}\quad\quad \mu(x) = \exp{\int P(x)\dd{x}}
\]




\Holine
\section*{Homogenous ODEs}
\[
    y'' + P(x)y' + Q(x)y = 0
\]
\textbf{Properties}
\begin{itemize}
    \item Linear combination of solutions is also a solution
    \item General solution on form $y(x) = c_1 y_1(x) + c_2 y_2(x)$
    \item Linearly independent solutions have a Wronskian of 0.
\end{itemize}
\[
    W(x) = 
    \begin{vmatrix} y_1(x) & y_2(x) \\ y_1'(x) & y_2'(x) \end{vmatrix}
\]


\holine
\subsection*{Variation of the constant}
If you have one of the two linearly indepedent solutions, you can find the other as $y_2(x) = C(x)\cdot y_1(x)$, where $C(x)$ is a functions determined by inserting $y_2(x)$ into the ODE.

When you arrive at a solution for $C(x)$, you may discard any constants or coefficients, i.e. $C(x) = \alpha x^3 + \beta$.


\holine
\subsection*{Constant coefficients - Particular Equation}
\[
    y''(x) + ay'(x) + by(x) = 0
\]

Solve the particular equation
\[
    \lambda^2 + a\lambda + b = 0
\]
for $\lambda_1$ and $\lambda_2$.


\subsubsection*{Two, real roots}
\[
    y(x) = C_1\exp{\lambda_1 x} + C_2\exp{\lambda_2 x}
\]


\subsubsection*{One, real root}
\[
    y(x) = (C_1 + xC_2)\exp{\lambda x}
\]


\subsubsection*{Two, complex roots}
\begin{equation*}
\begin{split}
    y(x) = A\exp{\lambda_1 x} + B\exp{\lambda_2 x} = \exp{-a/2 x}\qty[A\exp{i\omega x} + B\exp{-i\omega x}] \\
    =\exp{-a/2 x}\qty[\hat{A}\cos{\omega x} + \hat{B}\sin{\omega x}]
\end{split}
\end{equation*}


\holine
\subsection*{Euler-Cauchy Equations}
\[
    x^2y'' + a_1x y' + a_0 y = 0
\]

Introducing
\[
    x = \exp{z} \quad\quad\Rightarrow\quad\quad z = \ln|x|
\]
The equation can be rewritten to
\[
    \pdv[2]{y}{z} + (a_1-1)\pdv{y}{z} + a_0 y = 0
\]
Solve and insert for $z$.



\holine
\subsection*{Power methods}
\begin{itemize}
    \item Represent $P(x)$ and $Q(x)$ as power series.
    \item Assume solution on the form
    \begin{itemize}
        \item $y(x) = \sum_{n=0}^\infty a_n x^n$
        \item $y'(x) = \sum_{n=1}^\infty n a_n x^{n-1}$
        \item $y''(x) = \sum_{n=2}^\infty n(n-1) a_n x^{n-2}$ 
    \end{itemize}
    \item Insert back into ODE.
    \item Split into equations of matching powers of $x$.
\end{itemize}
This will give you one or two undetermined coefficients. The equations maybe give the coefficients as a series depending on each other, like $a_{s+1} = a_s^2$. If the series are odd/even, two undetermined coefficients are required to describe them, so the solution is complete.



\holine
\subsection*{Fröbenius method}
\[
    x^2 y'' + xb(x)y' + c(x) y = 0
\]
Assuming solution on form
\[
    y_p(x) = \sum_{m=0}^\infty a_m x^{m+s}
\]
where $s$ is some real number determined  the \textit{indicial equation}
\[
    s(s-1) + b_0s + c_0 = 0
\]
where $b_0=b(0)$, and $c_0=c(0)$

\textbf{Three possible scenarios:}
\begin{itemize}
    \item Different roots, $s_1\neq s_2$, and $s_1-s_2\neq \text{integer}$.
    \begin{itemize}
        \item Two indepedens solutions $y_i(x) = x^{si}\sum_{m=0}^\infty a_0 x^m$
    \end{itemize}
    \item Different roots, $s_1\neq s_2$, but $s_1-s_2=\text{integer}$. ($s_1>s_2$).
    \begin{itemize}
        \item Solve for both by Power Series.
        \item Often, $s_2$ gives the complete solution (two undetermined coefficients), so try this first.
        \item Sometimes, only $s_1$ gives a solution. Find the other by variation of the constant.
    \end{itemize}
    \item Double root, $s_1 = s_2$.
    \begin{itemize}
        \item Find the solution by Power Series.
        \item Find the second solution by variation of the constant.
    \end{itemize}
\end{itemize}



\Holine
\section*{Inhomogenous ODEs}
\[
    y'' + P(x)y' + Q(x)y = R(x)
\]
\textbf{Remember} to always rewrite to this form.

\textbf{Properties}
\begin{itemize}
    \item Solutions on form\\
    $y(x) = y_h(x) + y_p(x) = c_1y_1(x) + c_2y_2(x) + y_p(x)$.
    \item $y_h(x)$ is the solution to the homogenous equation.
    \item $y_p(x)$ is \textit{any} solution to the whole ODE.
    \item Since $y_h$ contains two arbritrary constants, $y_p$ should contain none.
\end{itemize}



\holine
\subsection*{Inhomo ODEs with constant coefficients}
\[
    y'' + ay' + by = R(x)
\]
\begin{itemize}
    \item Make a guess at $y_p$ with the same form as $R(x)$, with unknown coefficients.
    \item Insert back into ODE to solve for coefficients.
\end{itemize}
\textbf{Special case:} $R(x) = A\exp{kx}$.\\
Let $\alpha$ and $\beta $ be the roots of $\lambda^2 + a\lambda + b = 0$.
\begin{enumerate}
    \item If $k\neq \alpha,\beta$: Try $y_p = C\exp{\lambda x}$.
    \item If $k = \alpha \text{ or } \beta$: Try $y_p = Cx\exp{\lambda x}$.
    \item If $k = \alpha = \beta$: Try $y_p = C\exp{\lambda x}$.
\end{enumerate}


\holine
\subsection*{Inhomo ODEs with varying coefficients}
\[
    y'' + P(x)y' + Q(x)y = R(x)
\]

\subsubsection*{Factorization}
If $u(x)$ is a known solution to the homo-ODE, a particular solution is $y_p = u(x)\cdot v(x)$ where
\[
    w' = v \quad\quad\quad w' + \qty[\frac{2u'}{u} + P]w = \frac{R}{u}
\]
Solve the ODE for $w$ with integrating factor.


\subsubsection*{Variation of parameters}
\[
    y_p = -y_1\int\frac{y_2R}{W}\dd{x} + y_2 \int\frac{y_1R}{W}\dd{x}
\]
where $y_1$ and $y_2$ are known linearly indepedendent solutions to the homo-ODE.
\textbf{NOTE:} Remember that $R(x)$ is the RHS after the ODE is rewritten on the standard form.








%███████╗ ██████╗ ██╗   ██╗██████╗ ██╗███████╗██████╗ 
%██╔════╝██╔═══██╗██║   ██║██╔══██╗██║██╔════╝██╔══██╗
%█████╗  ██║   ██║██║   ██║██████╔╝██║█████╗  ██████╔╝
%██╔══╝  ██║   ██║██║   ██║██╔══██╗██║██╔══╝  ██╔══██╗
%██║     ╚██████╔╝╚██████╔╝██║  ██║██║███████╗██║  ██║
%╚═╝      ╚═════╝  ╚═════╝ ╚═╝  ╚═╝╚═╝╚══════╝╚═╝  ╚═╝
\newpage

\section*{Trigonometric Functions}
\subsection*{Usefull Shit}
\begin{itemize}
    \item recognize \textbf{odd} and \textbf{even} integrands. I.e $\infint \sin{x}/x^2 = 0$ due to odd, and $\infint\cos{x}/(1+x^2) = 2\zeroinfint\cos{x}/(1+x^2)$ due to even.
\end{itemize}

\subsection*{Orthogonality}
\[
    \int_{-\pi}^\pi \sin(mx)\sin(nx) \dd{x} = \int_{-\pi}^\pi \cos(mx)\cos(nx) \dd{x} = \pi \delta_{mn}
\]


\holine
\section*{Fourier Series}
\[
    f(x) = \half a_0 + \oneinfsum a_n\cos(\frac{n\pi x}{L}) + \oneinfsum a_n\sin(\frac{n\pi x}{L})
\]
\[
    a_n = \frac{1}{L}\int\limits_{-L}^{L}f(x) \cos(\frac{n\pi x}{L}) \dd{x} \quad\quad
    b_n = \frac{1}{L}\int\limits_{-L}^{L}f(x) \sin(\frac{n\pi x}{L}) \dd{x}
\]
\[
    f(x) = \sum_{n=-\infty}^{\infty} c_n \exp{in\pi x/L}  \quad\quad
    c_n = \frac{1}{2L}\int_{-L}^L f(x) \exp{-in\pi x/L}
\]

\subsection*{Even and Odd functions}
If $f(x)$ is \b{even} $\qty[f(x) = f(-x)]$:
\[
    a_n = \frac{2}{L}\int_0^L f(x)\cos(\frac{n\pi x}{L}) \dd{x} \quad\quad b_n = 0
\]
If $f(x)$ is \b{odd} $\qty[f(x) = -f(-x)]$:
\[
    a_n = 0 \quad\quad b_n = \frac{2}{L}\int_0^L f(x)\sin(\frac{n\pi x}{L}) \dd{x}
\]


\holine 
\subsection*{Dirichlet Conditions for Fourier Series}
\begin{enumerate}
    \item Finite number of min/max in interval.
    \item Finitite number of (only) finite discontinuities.
\end{enumerate}
If this holds, then the series will converge to $f(x)$ at all points. At discontinuities, the series will converge to the mid-point.
\holine
\subsection*{Parseval's Theorem}
\[
    \int_{-L}^L |f(x)|^2 \dd{x} = 2L\sum_{-\infty}^\infty |c_n|^2
\]



\holine
\section*{Fourier Transforms}
\[\yl{
    f(x) = \frac{1}{\sqrt{2\pi}} \infint F(k) \exp{ikx} \dd{k} \quad\quad
    F(k) = \frac{1}{\sqrt{2\pi}} \infint f(x) \exp{-ikx} \dd{x}
}\]

\subsection*{Odd and even functions}
If $f(x)$ is an odd function, $f(x) = -f(-x)$, the Fourier transform can be done using only sine (as cosine is symmetric around 0):
\[
    f(x) = \sqrt{\frac{2}{\pi}} i\int\limits_0^{\infty} F(k) \sin(k x) \dd{k}   \quad
    F(k) = \sqrt{\frac{2}{\pi}} i\int\limits_0^{\infty} f(x) \sin(k x) \dd{x}
\]
If $f(x)$ is even, $f(x) = f(-x)$, we need only cosine (as sine is anti-symmentric):
\[
    f(x) = \sqrt{\frac{2}{\pi}} \int\limits_0^{\infty} F(k) \cos(k x) \dd{k}   \quad
    F(k) = \sqrt{\frac{2}{\pi}} \int\limits_0^{\infty} f(x) \cos(k x) \dd{x}
\]


\subsection*{FT of a derivative}
\[
    \mathcal{F}\qty[f^{(n)}(x)] = (ik)^n \mathcal{F}\qty[f(x)]
\]


    




%██████╗ ██████╗ ███████╗
%██╔══██╗██╔══██╗██╔════╝
%██████╔╝██║  ██║█████╗  
%██╔═══╝ ██║  ██║██╔══╝  
%██║     ██████╔╝███████╗
%╚═╝     ╚═════╝ ╚══════╝
\newpage
\section*{Partial Differential Equations}
\subsection*{Notes}
\begin{itemize}
    \item In symentrical systems, it seems your can switch x<->y if it is required to suit boundary conditions (example: Diritchlet conditions are at x=a instad of at y=b).
    \item When resulting in cos/sin solutions of frequencies, include n=0 for cos, as it gives a non-zero solution, but not for sin.
\end{itemize}

\holine
\subsection*{Separation of variables}
1) The the solution function as a product of functions of each variable, i.e:
\[
    u(x,y) = X(x)Y(y) \quad\quad\quad u(r, \theta) = R(r)T(\theta)
\]
2) Insert this into the equation, and separate the equation into parts of only each variable. Each side must then be constant, and equals some \textit{separation constant}.

3) Solve each side of the equation (equaling the seperation constant), giving an infinite set of \textit{eigenfunctions}, $u_n(x,y)$ for the equation.

4) The final solution is a linear combination of the eigenfunctions.
\[
u(x,y) = \sum_{n=-\infty}^\infty c_n u_n(x,y)
\]



\holine
\subsection*{Laplace Equation - 2D Cartesian}
\[
    \nabla u(x,y) = 0
\]

Separation of variables, $u(x,y) = X(x)Y(y)$ gives solutions
\[
    u(x,y) = X(x)Y(y) = \qty{\begin{matrix} \exp{ky} \\ \exp{-ky} \end{matrix}} \times \qty{\begin{matrix} \sin(kx) \\ \cos(kx) \end{matrix}}
\]

\textbf{Diritchlet BC:} $u(x,0) = u(0,y) = u(a,y) = 0$, $u(x,b) = f(x)$
Eigenfunctions on form
\[
    u_n(x,y) = A_nF_n(x)G_n(x) = A_n\sin(n\pi x/a)\sinh(n\pi y/a)
\]


\holine
\subsection*{Wave Equation}
\[
    \pdv[2]{y}{x} = \frac{1}{c^2}\pdv[2]{y}{t}
\]
Seperation of variables, $u(x,t) = F(x)G(t)$ gives equations
\[
    F''(x) = -k^2F(x) \quad\quad\quad \ddot{G}(t) = -k^2c^2G(t)
\]
where the seperation constant, $-k^2$ must be negative, else the solutions would blow up.

The equations has solutions on the form
\[
    y(x,t) = \qty{\begin{matrix} \sin(kx) \\ \cos(kx) \end{matrix}} \times \qty{\begin{matrix} \sin(kct) \\ \cos(kct) \end{matrix}}
\]
The end-points are usually fixed at 0, leaving only the $\sin(kt)$ term, and forcing $k=n\pi/L$.

If the velocity is 0 at $t=0$, we discard the sin-velocity term and get

\[
    y(x,t) = \oneinfsum b_n\sin(\frac{n\pi x}{L})\cos(\frac{n\pi c t}{L})
\]

If position is 0 at $t=0$, we discard the cos-velocity term instead.

Initial position will be on the form
\[
    y(x,0) = \oneinfsum b_n \sin(\frac{n\pi x}{L}) = f(x)
\]
where $f(x)$ is the initial position or velocity. The coefficients $b_n$ are now Fourier coefficients, given as:
\[
    b_n = \frac{1}{2L}\int_{0}^{L} f(x)\sin(\frac{n\pi x}{L}) \dd{x}
\]



\holine
\subsection*{Orthogonal Functions}
Functions on the form
\[
    p(x)y'' + p'(x)y' + \qty[q(x) + \lambda r(x)]y = 0
\]
on some interval $[a,b]$ has solutions as linear combinations of eigenfunctions $y_n(x)$ which are orthogonal with respect to $r(x)$ such that
\[
    \int_a^b r(x)y_n(x)y_m(x)^* \dd{x} = 0 \quad\quad \text{for} \lambda_n \neq \lambda{m}
\]
Any function can be written as a linear combination of these eigenfunctions
\[
    f(x) = \sum_{n=1}^\infty a_n y_n(x)
\]
then the set $\{y_n(x)\}$ is complete. The coefficients $a_n$ are determined by the orthogonality:
\[
    a_n = \int_a^b f(x)r(x)y_n(x)^* \dd{x}
\]


\newpage





\end{multicols}
\end{document}
